\documentclass{article}
%\documentclass[preprint,12pt]{elsarticle}
%\journal{Computational Physics}
%\documentclass{article}
%\usepackage{graphicx}
%\documentclass[draft,jgrga]{agutex}
%\documentclass{jgi}

\usepackage{graphpap,natbib}
\usepackage{amssymb,epsfig,amsmath}
\newcommand{\VB}{\begin{verbatim}}
\newcommand{\VE}{\end{verbatim}}
\newcommand{\cov}[1]{\stackrel{\triangle}{#1}}
\newcommand{\con}[1]{\stackrel{\nabla}{#1}}
\newcommand{\nab}[2]{\nabla_{\displaystyle{#1}}#2}
\newcommand{\Lie}[2]{{\mathcal{L}}_{\displaystyle{#1}}#2}
\newcommand{\Riem}{\mathrm{Riem}}
\newcommand{\bs}[1]{{\boldsymbol{#1}}}
\newcommand{\de}{\,\mathrm d}
\newcommand{\D}[1]{\,\frac{\mathrm {D}{\,#1}}{\mathrm {D}t}}
\newcommand{\DD}[1]{\frac{\mathrm {D}^{2}{#1}}{\mathrm {D}{t}^{2}}}
\newcommand{\diff}[2]{\frac{\de{#1}}{\de{#2}}}
\newcommand{\diffn}[3]{\frac{\de^{#3}{#1}}{\de{#2}^{#3}}}

\newcommand{\PPar}[2]{\frac{\partial{#1}}{\partial{#2}}} 
\newcommand{\Dar}[2]{\frac{\delta{#1}}{\delta{#2}}} 

\newcommand{\Par}[2]{\displaystyle\frac{\partial{#1}}{\partial{#2}}}
\newcommand{\Parn}[3]{\frac{\partial^{#3}{#1}}{\partial{#2}^{#3}}}
\newcommand{\ha}{\tfrac{1}{2}}
\newcommand{\ds}{\displaystyle}
\renewcommand{\vec}[1]{\mbox{\boldmath{$#1$}}}
\newcommand{\tr}[1]{\mathrm{tr}(#1)}
\newcommand{\X}[1]{{\bf X}_{#1}}
\renewcommand{\(}{\left(}
\renewcommand{\)}{\right)}
\renewcommand{\[}{\left[}
\renewcommand{\]}{\right]}
\renewcommand{\t}{\widetilde}
\renewcommand{\r}[1]{(\ref{#1})}
\renewcommand{\O}[1]{{\cal O}(#1)}
\renewcommand{\^}{\wedge}
\renewcommand{\b}[1]{{\bf{#1}}}
\newcommand{\bv}[1]{\b{\bar{#1}}}
\renewcommand{\S}{(\nabla\b u+\nabla\b u^T)}
\newcommand{\nl}{\nonumber\\}
\newcommand{\gdot}{\dot\gamma}
\newcommand{\up}[1]{${}^{#1}$}
\newcommand{\ee}{\times 10}
%\newcommand{\X2}{{\bf X}_2}
%\usepackage{amssymb}

\setlength{\textwidth}{18cm} \setlength{\oddsidemargin}{-1cm}
\setlength{\evensidemargin}{.5cm} \setlength{\footskip}{1cm}
\setlength{\textheight}{25cm}\setlength{\topmargin}{-2cm}

\begin{document}
\title{Paramesh\\Anisotropic calculations 2D and 3D}
\maketitle
\section{Anisotropic calculations}
Given
\begin{align*}
 G=\int g(\nabla\phi)\equiv\int\ha A^2|\nabla\phi|^2
\end{align*}
First we assume the identity
\begin{align*}
 -\Dar{G}{\phi}=\nabla\cdot\Par{g}{\nabla\phi}\equiv \phi_{,ij}\Par{}{\phi_i}\Par{g}{\phi_j}
\end{align*}
where 
\begin{align*}
 \phi_{,ij}\equiv\Par{{}^2\phi}{x^i\partial x^j}
\end{align*}
We also denote (without comma)
\begin{align*}
g_i\equiv\Par{g}{\phi_i},\quad g_{ij}\equiv \Par{{}^2g}{\phi_i\partial\phi_j}
\end{align*}
so that
\begin{align*}
 g_i=AA_i|\nabla\phi|^2+A^2\phi_i
\end{align*}
and
\begin{align*}
 g_{ij}=A_iA_j|\nabla\phi|^2+2A(A_i\phi_j+A_j\phi_i)+A|\nabla\phi|^2A_{ij}+A^2\phi_{ij}
\end{align*}
where we use the notation

\subsection{2 D}
Given 
\begin{align*}
 A\equiv A_0\[1+\epsilon\(\frac{\phi_{,1}^4+\phi_{,2}^4}{|\nabla\phi|^4}\)\]\equiv A_0\[1+\epsilon\(1-\frac{2\phi_{,1}^2\phi_{,2}^2}{|\nabla\phi|^4}\)\]
\end{align*}
we find using $X=(\phi_{,x}/|\nabla\phi|)^2,Y=(\phi_{,y}/|\nabla\phi|)^2$ and $A_i\equiv\Par{A}{\phi_{,,i}}$, that
\begin{align*}
 A_1=4\,{\frac {{\it A_0}\,\epsilon\,\phi_{,1}{Y} \left( {X}-{Y}
 \right) }{ |\nabla\phi|^2}}
\end{align*}
\begin{align*}
 A_2=4\,{\frac {{\it A_0}\,\epsilon\,{X}\phi_{,2} \left( {Y}-{X}
 \right) }{ |\nabla\phi|^2}}
\end{align*}
\begin{align*}
 A_{11}=-4\,{\frac {{\it A_0}\,\epsilon\,{Y} \left( 3\,{X^2}-8\,{X}{Y}+{Y^2} \right) }{ |\nabla\phi|^2}}
\end{align*}
\begin{align*}
 A_{22}=-4\,{\frac {{\it A_0}\,\epsilon\,{X} \left( {X^2}-8\,{X}{Y}+3\,{Y^2} \right) }{ |\nabla\phi|^2}}
\end{align*}
\begin{align*}
A_{12}= 8\,{\frac {{\it A_0}\,\epsilon\,\phi_{,1}\phi_{,2} \left( {X^2}-4\,{X}{Y}+Y^2 \right) }{|\nabla\phi|^4}}
\end{align*}
To avoid floating point errors define
\begin{align*}
 \tilde A_i=A_i|\nabla\phi|^2\text{ and } \tilde A_{ij}=A_{ij}|\nabla\phi|^2
\end{align*}
so that
\begin{align*}
 g_{ij}=\frac{1}{|\nabla\phi|^2}\[\tilde A_i\tilde A_j+2A(\tilde A_i\phi_{,j}+\tilde A_j\phi_{,i})\]+A\tilde A_{ij}+A^2\delta_{ij}
\end{align*}
\subsection{3D}
\begin{align*}
 A&\equiv A_0\[1+\epsilon\(\frac{\phi_1^4+\phi_2^4+\phi_3^4}{|\nabla\phi|^4}\)\]\nl
 A_1&=\frac{4A_0\epsilon\phi_{,1}}{|\nabla\phi|^2}(X(Y+Z)-Y^2-Z^2)\nl
A_{12}&=\frac{8A_0\epsilon\phi_{,1}\phi_{,2}}{|\nabla\phi|^4}(X^2+Y^2-4XY-2Z(X+Y)+3Z^2)\nl
A_{11}&=\frac{-4A_0\epsilon}{|\nabla\phi|^2}(3X^2(Y+Z)-8X(Y^2+Z^2)-6XYZ+Y^3+Y^2Z+Z^2Y+Z^3)
\end{align*}
and we can find the other $A_{ij}$ by cyclic permutation.
\subsection{Isolating the Laplacian}
in 2D
\begin{align*}
 \phi_{,ij}g_{ij}&=g_{11}\phi_{,11}+g_{22}\phi_{,22}+2g_{12}\phi_{,12}\nl
&=\ha(g_{11}+g_{22})(\phi_{,11}+\phi_{,22})+\ha(g_{11}-g_{22})(\phi_{,11}-\phi_{,22})+2g_{12}\phi_{,12}\nl
&=\ha(g_{11}+g_{22})\nabla^2\phi+\ha(g_{11}-g_{22})(\phi_{,11}-\phi_{,22})+2g_{12}\phi_{,12}\nl
\end{align*}
For the purposes of the pointwise Newton-Raphson in the multigrid smoother, the functions, $g_{ij}$, are discretised without using the central node. So the only contributions to the Newton method are from the discretisation of the Laplacian. Example, the 5 point stencil has a contribution from the central node, $-4/\Delta x^2$, giving net contribution to the smoother of
\begin{align*}
 -\ha(g_{11}+g_{22})4/\Delta x^2
\end{align*}
Moreover, different stencils may be used for the Laplacian making it advantageous to isolate this term.
 
The procedure above is equivalent to writing the matrix $\b g$ as
\begin{align*}
 \b g=\ha \tr{\b g}\b I+\tilde{\b g},
\end{align*}
where $\b I$ is the identity, so that
\begin{align*}
 \tr{\tilde{\b g}}=0
\end{align*}
and taking advantage of $I_{ij}\phi_{ij}=\nabla^2\phi$.
Similarly in 3D
\begin{align*}
 \b g=\tfrac{1}{3} \tr{\b g}\b I+\tilde{\b g}
\end{align*}
guarantees $\tr{\tilde{\b g}}=0$ and thus
\begin{align*}
 \phi_{,ij}g_{ij}&=\tfrac{1}{3}\tr{\b g}\nabla^2\phi+\phi_{,ij}\tilde g_{ij}
\end{align*}


\end{document}
